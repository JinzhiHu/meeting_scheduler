\documentclass{article}
\usepackage{graphicx} % Required for inserting images
\usepackage{amsmath,amssymb}
\usepackage{blkarray}
\usepackage{bbm}
\usepackage{optidef}
\usepackage{pdfpages}
\usepackage{titling}
\usepackage{float}
\usepackage{tikz}
%\usepackage{hyperref}
\usepackage{listings}
\usepackage{pdfpages}
\usepackage[dvipsnames]{xcolor}
\usepackage{svg}
\usepackage[left=1in, right=1in, top=1.6in, bottom=1.6in]{geometry} %%

\begin{document}
\begin{center}
{\huge Project Explained
}
\end{center}
Recall the bus problem in Minimum Network flow (Min-flow) problem, we constructed a network representing the bus route, and when minimized, the arch $i$-$j$ represents the number of tickets sold under the demand for passenger onboarding the bus in $i$ heading to destination $j$.

The arches in the bottom of the network ensures the bus is not overloaded with an upper bound $P$. Also, by the nature of the Min-flow problem, fares for bus rides are multiplied by $-1$.

The graph below represents the minimizing min-flow problem after conversion, where fares (non-zero values) are represented in {\color{ForestGreen} green}, finite upperbonds on arches are coloured in  {\color{blue} blue}, and the differences between inflow and outflow on vertices are in black. Without specification, all other parameters are unrestricted.

\begin{center}
\includegraphics[width=5in]{bus_network.pdf}

{\small \textbf{Fig 1}: Bus problem}
\end{center}

We observed that a meeting schedule is similar to a bus service with only one seat ($P=1$), where the attendee is selecting between meetings that resembles selecting a passenger that ``rides" on a one-seater ``bus."

In the modified problem, we define bottom vertices as the start and end times of meetings. We define $f_{ij}$ as the values of the meeting from time $i$ to time $j$. Here is the modified Bus problem for a meeting schedule optimization program.

\begin{center}
\includegraphics[width=5in]{schedule_network.pdf}

{\small \textbf{Fig 2}: Schedule problem}
\end{center}
\end{document}